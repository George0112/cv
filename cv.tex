%%%%%%%%%%%%%%%%%%%%%%%%%%%%%%%%%%%%%%%%%%%%%%%%%%%%%%%%%%%%%%%%%%%%%%%%%%%%%%%%
% Medium Length Graduate Curriculum Vitae
% LaTeX Template
% Version 1.2 (3/28/15)
%
% This template has been downloaded from:
% http://www.LaTeXTemplates.com
%
% Original author:
% Rensselaer Polytechnic Institute 
% (http://www.rpi.edu/dept/arc/training/latex/resumes/)
%
% Modified by:
% Daniel L Marks <xleafr@gmail.com> 3/28/2015
%
% Important note:
% This template requires the res.cls file to be in the same directory as the
% .tex file. The res.cls file provides the resume style used for structuring the
% document.
%
%%%%%%%%%%%%%%%%%%%%%%%%%%%%%%%%%%%%%%%%%%%%%%%%%%%%%%%%%%%%%%%%%%%%%%%%%%%%%%%%

%-------------------------------------------------------------------------------
%	PACKAGES AND OTHER DOCUMENT CONFIGURATIONS
%-------------------------------------------------------------------------------

%%%%%%%%%%%%%%%%%%%%%%%%%%%%%%%%%%%%%%%%%%%%%%%%%%%%%%%%%%%%%%%%%%%%%%%%%%%%%%%%
% You can have multiple style options the legal options ones are:
%
%   centered:	the name and address are centered at the top of the page 
%				(default)
%
%   line:		the name is the left with a horizontal line then the address to
%				the right
%
%   overlapped:	the section titles overlap the body text (default)
%
%   margin:		the section titles are to the left of the body text
%		
%   11pt:		use 11 point fonts instead of 10 point fonts
%
%   12pt:		use 12 point fonts instead of 10 point fonts
%
%%%%%%%%%%%%%%%%%%%%%%%%%%%%%%%%%%%%%%%%%%%%%%%%%%%%%%%%%%%%%%%%%%%%%%%%%%%%%%%%

\documentclass[margin]{res}  

% Default font is the helvetica postscript font
\usepackage{helvet}

% Increase text height
\textheight=700pt

\begin{document}

%-------------------------------------------------------------------------------
%	NAME AND ADDRESS SECTION
%-------------------------------------------------------------------------------
\name{Chao Wen Chen}

% Note that addresses can be used for other contact information:
% -phone numbers
% -email addresses
% -linked-in profile

\address{\\LinkedIn : https://www.linkedin.com/in/chao-wen-chen-5a0a61169/
\\Github : https://github.com/George0112\\}
\address{\\chaowen.nthu@gmail.com \\(+886) 931875878}

% Uncomment to add a third address
%\address{Address 3 line 1\\Address 3 line 2\\Address 3 line 3}
%-------------------------------------------------------------------------------

\begin{resume}

%-------------------------------------------------------------------------------
%	EDUCATION SECTION
%-------------------------------------------------------------------------------
\section{OBJECTIVE}
{\sl Senior student and ready for the master program both in NTHU. Inquisitive, hard-working and consistent. Looking for internship opportunities at Skymizer where I can apply my skills and contribute to real-world projects. }

\section{EDUCATION}
\textbf{National Tsing Hua University}, Taiwan\\
{\sl Bachelor Engineering}, Computer Science and Engineering(CSE)\\
Expected June, 2019
%\hfill CGPA: 7.34/10.00\\

% \textbf{Zhong Li High School}, Taiwan\\
% {\sl Special Class for Mathematic\\
% July 2015
%\hfill Aggregate 87.8\%

%\textbf{St. Joseph Public School}, Kota (Raj.) \\
%{\sl Class X (Secondary Examination)}, CBSE\\
%July 2013
%\hfill CGPA: 10/10



%-------------------------------------------------------------------------------
%	COMPUTER SKILLS SECTION
%-------------------------------------------------------------------------------
\section{TECHNICAL\\SKILLS}

\textbf{Languages : } Javascript, Python, C++, C, PHP
\\
\textbf{Database :} MySQL, PostgreSQL
\\
\textbf{Tools/Framework : } Laravel(PHP), Djanogo(Python), Node.js
\\
\textbf{Familiar : } PHP, Javascript, React, React-Native, HTML, CSS
\\
\textbf{General : } Data Structures, Algorithm, Object Oriented Programming

%-------------------------------------------------------------------------------
% Modify the format of each position
\begin{format}
\title{l}\\
\dates{l}\location{r}\\
\body\\
\end{format}
%-------------------------------------------------------------------------------

\section{EXPERIENCE}

\textbf{Website Administrator and Developer of ITRI \hfill{Jun 17 - Mar 19}\\}
\normalfont{We developed a website for paper searching and recommanding. More than 700 people have registered and more than 2000 paper is included in this website. }

\textbf{Research Assistant of NMSL@NTHU \hfill{Jun 17 - }\\}
\normalfont{Our research field covers network and all kinds of media. Now we are looking for a better solution for the next generation network. E.g. SDN, QUIC, AR, VR, etc.}

\textbf{Summer Intern of Administration Yuan \hfill{Jul 18 - Aug 18}\\}
\normalfont{It's a project for checking and revise websites of each departmanet in the gonvernment. I checked more than 100 website and give them advices to make it better.}


\section{PROJECTS}
%\employer{LNMIIT}
\location{}
\title{\textbf{Smart Manufacturing Digital Library \hfill Mar 2019}
 }
\begin{position}
% \begin{itemize}
A laravel based website includes thousands of papers with well arrangement and recommendation. We separated  front-end and back-end clearly with restful APIs.
% \end{itemize}
\begin{itemize}
\item \textbf{Technology/Tools:} PHP, Javascript, HTML, CSS
\end{itemize}
\begin{itemize}
\item \textbf{Link :} industry4.tw/
\end{itemize}
\end{position}

%\employer{LNMIIT}
\location{}
\title{\textbf{Streaming scalable video sequences with media-aware network elements implemented in P4 programming language \hfill Apr 2018}
 }
\begin{position}
% \begin{itemize}
Based on P4, a SDN language, to realize a media-aware network elements. We used bmv2 and SVEF to do some experiments. The result shows that we can maintain stable video streaming than regular switches.
% \end{itemize}
\begin{itemize}
\item \textbf{Technology/Tools:} C, P4, Python 
\end{itemize}
 \begin{itemize}
 \item \textbf{Link :} ieeexplore.ieee.org/abstract/document/8406129
 \end{itemize}
\end{position}


\section{Publications}
\par
\normalfont{}
\\
\normalfont{[1] G. Wang, \textbf{C. Chen}, C. Chen, L. Pan, Y. Wang, C. Fan, and C. Hsu, "Streaming scalable video sequences with media-aware network elements implemented in P4
programming language," in Proc. of IEEE Network Operations and Management Symposium (NOMS’18) Demo Session, Taipei, Taiwan, April 2018, pp. 1–2.}

\\

\normalfont{[2] Q. Zhu, M. Uddin, N. Venkatasubramanian, \textbf{C. Chen}, and C. Hsu, "Spatio-temporal scheduling of in-Situ and mobile Internet-of-Things devices for urban
environment sensing," ACM Transactions on Internet of Things, in preparation, October 2018.}

% \section{Research Grant}{}
% \\
% \normalfont{[1] Streaming scalable video Sequences with media-aware network elements, MOST Undergraduate Research Project (#107-2813-C-007 -074 -E), \$48,000, July 2018 - February 2019}

% \\
% \section{Course Design}{}
% \\
% \normalfont{[1] Software Defined Network Practice in P4 and ONOS}


% \section{RELEVANT\\COURSES}
% \par

% \normalfont{\textbullet{} Introduction to Data Science }
% \normalfont{ \textbullet{} Artificial Intelligence}
% \normalfont{ \textbullet{} Software Engineering \\}
% \normalfont{\textbullet{} Multimedia Processing and Application   }
% \normalfont{\textbullet{} Computer Graphics\\}
% \normalfont{\textbullet{} Data Structures and Algorithm }
% \normalfont{ \textbullet{} Introduction to Database Management \\}
% \normalfont{\textbullet{} Advanced Programming in Java}
% \normalfont{\textbullet{} Operating System}
% \normalfont{\textbullet{} Computer Networks}

% \section{ADDITIONAL ACTIVITIES}

% \normalfont{ \textbullet{} Member of Computer Society of India, LNMIIT Chapter \\}
% \normalfont{ \textbullet{} Attended DevFest-GDG Jaipur meetup, 2018 \\}
% \normalfont{\textbullet{} Volunteering experience at WikiToLearn India conference, 2017 \\ }

%-------------------------------------------------------------------------------


\end{resume}
\(\)\end{document}